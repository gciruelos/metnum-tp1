\subsection{Demo}
Primero observemos que A es una matriz diagonal dominante no estricta. Para eso tenemos que ver que para cada fila el valor absoluto de la diagonal es mayor o igual que la norma-1 del resto de los elementos de esa fila. En nuestro caso puntual, esto significa ver que 
$|c_k| \geq |a_k| + 2|b_k| + |d_k|$\footnote{Los coeficientes están definidos en la sección \ref{sec:armado-sistema}}.

Primero calculemos el lado derecho de la desigualdad:
\begin{equation*}
\left\vert \dfrac{r_k - \Delta r}{r_k (\Delta r)^2}\right\vert +
2 \times \left\vert \dfrac{1}{r_k^2(\Delta \theta)^2} \right\vert+
\left\vert \dfrac{1}{(\Delta r)^2} \right\vert = 
\dfrac{\left\vert r_k - \Delta r \right\vert}{r_k (\Delta r)^2} +
2 \times \dfrac{1}{r_k^2(\Delta \theta)^2} +
\dfrac{1}{(\Delta r)^2}
\end{equation*}

Asumamos primero que $\left\vert r_k - \Delta r \right\vert \geq 0$. En ese caso la ecuación anterior es igual a
 
\begin{center}
$\dfrac{2 r_k^2 (\Delta \theta)^2 - (\Delta r) r_k (\Delta \theta)^2 + 2 (\Delta r)^2}{r_k^2 (\Delta r)^2 (\Delta \theta)^2}$
\end{center}

Entonces, queremos ver que $|b_k|$ es mayor o igual que eso, es decir

\begin{equation*}
\dfrac{\left\vert -2 r_k^2 (\Delta \theta)^2 + (\Delta r) r_k (\Delta \theta)^2 - 2 (\Delta r)^2 \right\vert } 
{\left\vert r_k^2 (\Delta r)^2 (\Delta \theta)^2 \right\vert } \geq
\dfrac{2 r_k^2 (\Delta \theta)^2 - (\Delta r) r_k (\Delta \theta)^2 + 2 (\Delta r)^2}{r_k^2 (\Delta r)^2 (\Delta \theta)^2}
\end{equation*}

Que es equivalente a

\begin{equation*}
\left\vert -2 r_k^2 (\Delta \theta)^2 + (\Delta r) r_k (\Delta \theta)^2 - 2 (\Delta r)^2 \right\vert \geq
2 r_k^2 (\Delta \theta)^2 - (\Delta r) r_k (\Delta \theta)^2 + 2 (\Delta r)^2
\end{equation*}

Supongamos que lo que está dentro del módulo es positivo, entonces tenemos

\begin{center}
$-2 r_k^2 (\Delta \theta)^2 + (\Delta r) r_k (\Delta \theta)^2 - 2 (\Delta r)^2 \geq
2 r_k^2 (\Delta \theta)^2 - (\Delta r) r_k (\Delta \theta)^2 + 2 (\Delta r)^2$ \\
$\Updownarrow$\\
$2\times (-2 r_k^2 (\Delta \theta)^2 + (\Delta r) r_k (\Delta \theta)^2 - 2 (\Delta r)^2) \geq 0$\\
$\Updownarrow$\\
$-2 r_k^2 (\Delta \theta)^2 + (\Delta r) r_k (\Delta \theta)^2 - 2 (\Delta r)^2 \geq 0$
\end{center}

Pero esta última desigualdad vale pues habíamos supuesto que efectivamente eso era positivo.

Ahora veamos que pasa si lo de adentro del módulo es negativo. Tenemos que 

\begin{center}
$-2 r_k^2 (\Delta \theta)^2 + (\Delta r) r_k (\Delta \theta)^2 - 2 (\Delta r)^2 \leq
-2 r_k^2 (\Delta \theta)^2 + (\Delta r) r_k (\Delta \theta)^2 - 2 (\Delta r)^2$ \\
$\Updownarrow$\\
$0 \leq 0$\\
\end{center}

Que vale trivialmente. Luego probamos que la matriz del sistema es diagonal dominante no estricta, si $~{\left\vert r_k - \Delta r \right\vert \geq 0}$. Notar además que por la última cuenta no es cierto que sea diagonal dominante estricta.