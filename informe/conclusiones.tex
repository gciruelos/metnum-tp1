Para empezar, queremos decir que creemos que este trabajo práctico fue fructífero. Pudimos implementar correctamente los métodos numéricos de resolución de sistemas lineales vistos en clase, lo cual nos ayudó mucho a entenderlos e internalizarlos mucho más.

Además fue muy positiva la experiencia de utilizar estos métodos para resolver un problema de la vida real. Este TP, por otro lado, nos permitió darnos un poco de idea de cuáles son las aplicaciones más usuales de los algoritmos vistos en la materia. Otra cosa notable que se desprende de este TP es lo importante que es abarcar un problema interdisciplinariamente: en este caso necesitamos conocer la ecuación de Laplace (física), saber que una derivada se puede aproximar con un cociente incremental con diferencias finitas (matemática) y utilizar métodos numéricos para resolver sistemas de ecuaciones lineales (computación).

En la misma linea, tomar tiempos de los algoritmos y comparar sus performances nos permitió corroborar lo que ya sabíamos en la teoría de manera práctica, además de ganar intuición sobre el comportamiento de los algoritmos.

Por otro lado, nos gustaría plantear trabajo a futuro, y atar algunos cabos sueltos. Nuestra implementación \emph{optimizada} de los algoritmos de eliminación gaussiana y factorización LU que aprovechaban el hecho de que la matriz es banda, solamente tenía una ganancia de performance con respecto al tiempo, dado que seguimos almacenando la matriz en un vector de vectores.

Por esa razón, es interesante plantearse el problema de como llevar a cabo esta optimización de la complejidad espacial de nuestros métodos. Algunas ideas en lo que respecta a la solución de este problema pueden hallarse en \cite[Cap. 4.3]{golub}. 




