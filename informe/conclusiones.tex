Para empezar, queremos decir que creemos que este trabajo práctico fue fructífero. Pudimos implementar correctamente los métodos numéricos de resolución de sistemas lineales vistos en clase, lo cual nos ayudó mucho a entenderlos e internalizarlos mucho más.

Además fue muy positiva la experiencia de utilizar estos métodos para resolver un problema de la vida real. Este TP, por otro lado, nos permitió darnos un poco de idea de cuáles son las aplicaciones más usuales de los algoritmos vistos en la materia. Otra cosa notable que se desprende de este TP es lo importante que es abarcar un problema interdisciplinariamente: en este caso necesitamos conocer la ecuación de Laplace (física), saber que una derivada se puede aproximar con un cociente incremental con diferencias finitas (matemática) y utilizar métodos numéricos para resolver sistemas de ecuaciones lineales (computación).

En la misma línea, tomar tiempos de los algoritmos y comparar sus performances nos permitió corroborar lo que ya sabíamos en la teoría de manera práctica, además de ganar intuición sobre el comportamiento de los algoritmos.

Una cosa que nos gustaría decir, es que los experimentos que llegaron al TP son sólo una pequeña proporción de todos los que realizamos. Muchos quedaron afuera porque estaban mal hechos (pocas instancias de prueba, bugs) o simplemente porque no mostraban de forma tan clara lo que queriamos expresar. De manera iterativa, llegamos a los experimentos que presentamos que, desde nuestro punto de vista, expresan perfectamente lo que queríamos que expresaran.

Pasando a las conclusiones sobre la experimentación propiamente dicha, pudimos determinar empíricamente las siguientes cuestiones:
\begin{itemize}
\item Una mayor granularidad de los ángulos que se toman en la discretización favorece a detectar picos localizados de temperaturas que podrían pasar desapercibidos de lo contrario. A su vez, una mayor granularidad de los radios permite delinear con mucha mayor precisión la isoterma. Sin embargo, como vimos, a medida que aumentamos la cantidad de radios la posición de la isoterma va convergiendo a una posición que varía cada vez menos.
\item Aumentar la granularidad de la discretización aumenta también el tiempo de cómputo requerido para resolver el sistema de ecuaciones. Para el caso en que sólo se pasa una instancia, los métodos de eliminación gaussiana y de factorización LU muestran una performance casi idéntica, mientras que en el caso en que se toman múltiples instancias vemos como la factorización LU supera ampliamente a la eliminación gaussiana, confirmando lo que se esperaba desde la teoría (la complejidad de eliminación gaussiana es cúbica para resolver cada instancia, mientras que LU tiene un costo amortizado cuadrático). Por otro lado, las versiones optimizadas que aprovechan el hecho de que la matriz sea banda demostraron ser por mucho superiores a sus contrapartes tradicionales.
\item Por los dos puntos anteriores, es posible encontrar una granularidad \emph{óptima} (en algún sentido) tal que la isoterma obtenida sea lo suficientemente precisa sin pagar un costo en tiempo de cómputo excesivo, pues dijimos que al aumentar los radios la posición de la misma convergía, llegando un punto en el que un aumento de la granularidad produce una difertencia mínima en el resultado.
 \end{itemize}

Para finalizar, nos gustaría plantear trabajo a futuro, y atar algunos cabos sueltos. Nuestra implementación \emph{optimizada} de los algoritmos de eliminación gaussiana y factorización LU que aprovechaban el hecho de que la matriz es banda, solamente tenía una ganancia de performance con respecto al tiempo, dado que seguimos almacenando la matriz en un vector de vectores.

Por esa razón, es interesante plantearse el problema de como llevar a cabo esta optimización de la complejidad espacial de nuestros métodos. Algunas ideas en lo que respecta a la solución de este problema pueden hallarse en \cite[Cap. 4.3]{golub}. 




