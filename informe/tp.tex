\documentclass[hidelinks,a4paper,10pt, nofootinbib]{article}
\usepackage[width=15.5cm, left=3cm, top=2.5cm, right=2cm, left=2cm, height= 24.5cm]{geometry}
\usepackage[spanish]{babel}
\usepackage[utf8]{inputenc}
\usepackage[T1]{fontenc}
\usepackage{xspace}
\usepackage{xargs}
\usepackage{fancyhdr}
\usepackage{lastpage}
\usepackage{caratulaMetNum}
\usepackage[bottom]{footmisc}
\usepackage{amssymb}
\usepackage{algorithm}
\usepackage[noend]{algpseudocode}

\usepackage{graphicx}
\usepackage{sidecap}
\usepackage{amsmath}
\usepackage{wrapfig}
\usepackage{caption}

\usepackage{hyperref}
\hypersetup{
  colorlinks   = true, %Colours links instead of ugly boxes
  urlcolor     = blue, %Colour for external hyperlinks
  linkcolor    = blue, %Colour of internal links
  citecolor   = red %Colour of citations
}

\usepackage{comment}

\usepackage[
  backend=bibtex,
  style=alphabetic
]{biblatex}
\addbibresource{bibliografia.bib}


\captionsetup[table]{labelsep=space}


\setlength{\parindent}{4em}
\setlength{\parskip}{0.5em}


%%fancyhdr
\pagestyle{fancy}
\thispagestyle{fancy}
\addtolength{\headheight}{1pt}
\lhead{Métodos Numéricos: TP1}
\rhead{$2º$ cuatrimestre de 2015}
\cfoot{\thepage\ / \pageref{LastPage}}
\renewcommand{\footrulewidth}{0.4pt}

%%caratula
\materia{Laboratorio de Métodos Numéricos}
\titulo{Trabajo Práctico Número 1}
\subtitulo{Con 15 $\theta$'s discretizo alto horno}
%\grupo{Grupo 12}
\integrante{Ciruelos Rodríguez, Gonzalo}{063/14}{gonzalo.ciruelos@gmail.com}
\integrante{Costa, Manuel José Joaquín}{035/14}{manuc94@hotmail.com}
\integrante{Gatti, Mathias Nicolás}{477/14}{mathigatti@gmail.com}

\abstracto{En el presente trabajo práctico nos proponemos modelar un sistema complejo: la pared de un alto horno de fundición, para poder encontrar isotermas en la misma a partir de conocer las temperaturas interiores y exteriores. A este fin haremos uso de la ecuación diferencial del calor de Laplace, discretizando las derivadas a partir de diferencias finitas atrasadas y adelantadas, al tiempo que particionaremos la pared con ángulos y radios. Luego, podremos plantear un sistema de ecuaciones que nos permita obtener la temperatura en los puntos de la discretización, lo cual haremos usando los métodos de eliminación gaussiana y factorización LU. Notaremos que la matriz de este sistema posee características especiales que aseguran la existencia de factorización LU y permiten optimizar los métodos. Realizaremos experimentos para ver la relación entre la granularidad de la discretización y el tiempo de cómputo de los métodos, así como la proximidad de la isoterma buscada a la pared exterior. También contrastaremos la efectividad de los métodos usados entre si, en distintos casos. Finalmente concluiremos que se verifican las complejidades teóricas de los métodos númericos, además de que a medida que se aumenta la granularidad la posición de la isoterma tiende a estabilizarse en una posición como sería deseable.}

\palabraClave{Eliminación Gaussiana}
\palabraClave{Factorización LU}
\palabraClave{Discretización}
\palabraClave{Diferencia Finita}

\begin{document}
\maketitle
\section{Introducción teórica}
\input{intro.tex}
\newpage

\section{Desarrollo}
\subsection{Convenciones}
De aquí en adelante, si no se aclara otra cosa, se asumen las siguientes convenciones:
\begin{itemize}
\item$r_i$ es el radio que va del centro del horno al borde interno de la pared, mientras que $r_e$ es el radio considerando el borde externo;
\item$n$ es la cantidad de ángulos discretos ($0=\theta_0 <\hdots< \theta_{n-1} = 2\pi - \Delta \theta$) en los que se particiona la pared;
\item$m+1$ es el total de radios discretos ($~{r_i=r_0<\hdots<r_m=r_e}$); 
\item$t_{k,j} = T(r_k, \theta_j)$, donde $T$ es la función de temperatura de la pared (que desconocemos);
\item$b\in \mathbb{R}^{n\times (m+1)}: b = (b_0, b_1, \hdots, b_{(n\times (m+1))-1})$ será el vector de términos independientes del sistema que plantearemos luego (veremos que la dimensión escogida es correcta). 
\end{itemize}
\subsection{Métodos numéricos usados}
A partir de la ecuación del calor de Laplace y las discretizaciones de las derivadas parciales dadas en el enunciado del presente trabajo práctico, se puede obtener un sistema de ecuaciones donde las soluciones son las temperaturas en los puntos de la discretización. 
Es decir que el problema de hallar la isoterma se reduce a dos sub-problemas: el primero es resolver efectivamente el sistema planteado, mientras que el segundo consiste en usar las temperaturas halladas para estimar la posición de la isoterma.\\
Para resolver el sistema haremos uso de los métodos de eliminación gaussiana y la factorización LU, para luego poder contrastar su eficiencia ante distintas situaciones.\\
Para hallar la posición de la isoterma (de temperatura $t_{iso}$), dado un ángulo $\theta_j$ de la discretización, lo que haremos es buscar dos radios, $r_k$ y $r_{k+1}$, tales que $t_{k+1,j} \leq t_{iso} \leq t_{k, j}$. A partir de estos dos valores, facilmente podemos realizar un ajuste lineal considerando la recta de pendiente $m = \dfrac{t_{k,j}-t_{k+1,j}}{\Delta r}$ que pasa por $(r_k,t_{k,j})$ y $(r_{k+1}, t_{k+1,j})$. Para una explicación más detallada ver el apéndice ?????.

  
\subsection{Armado del sistema de ecuaciones y su matriz asociada}
Tenemos un sistema con $n\times(m+1)$ incógnitas.\\
Primero contamos con $n$ variables, los $t_{0,j}$ con $j = 0,1,\hdots, n-1$ (es decir, las temperaturas interiores), cuyos valores conocemos pues nos son dados como inputs. Lo mismo sucede con $t_{m, j}$ para $j = 0,1,..., n-1$ (temperaturas externas).
Luego, para las temperaturas interiores sabemos gracias a la función de Laplace y a la discretización de las derivadas que vale que \\

$\dfrac{t_{k-1,j} - 2t_{k,j} + t_{k+1,j}}{(\Delta r)^2} 
+ \dfrac{1}{r_k} \times \dfrac{t_{k,j} - t_{k-1,j}}{\Delta r}
+ \dfrac{1}{r_{k}^2} \times \dfrac{t_{k,j-1} -2t_{k,j} + t_{k,j+1}}{(\Delta \theta)^2} = 0$\\

Reescribiendo convenientemente la ecuación de arriba nos queda que \\

\begin{equation}
\label{laplace}
\dfrac{r_k - \Delta r}{r_k (\Delta r)^2} t_{k-1, j} +
\dfrac{r_k \Delta r (\Delta \theta)^2 - 2(\Delta r)^2}{r_k^2 (\Delta r)^2 (\Delta \theta)^2} t_{k,j} +
\dfrac{1}{(\Delta r)^2} t_{k+1,j} +
\dfrac{1}{r_k^2(\Delta \theta)^2} t_{k, j-1} +
\dfrac{1}{r_k^2(\Delta \theta)^2} t_{k, j+1} = 0
\end{equation}

Vale destacar que si $j = 0$ entonces en lugar de $t_{k, j-1}$ se usa $t_{k, n-1}$, mientras que si $j = n-1$ en lugar de $t_{k, j+1}$ va $t_{k, 0}$.\\
De esto se desprende inmediatamente que el valor de cada $t_{k,j}$ ($1\leq k < m$) depende exclusivamente de sus cuatro vecinos. Cabe mencionar también que si bien todos los coeficientes dependen de la distancia al centro del horno ($r_k$), ninguno lo hace respecto del ángulo concreto en que se encuentra el punto. Esto resulta muy razonable si tenemos en cuenta que el valor del ángulo depende exclusivamente del sistema de referencia escogido (dónde ubicamos el ángulo 0), mientras que dado un punto de la pared es lógico que su temperatura no dependa del sistema de referencia escogido para su medición.\\ 
Para facilitar la lectura, de ahora en adelante llamaremos a los coeficientes de la ecuación (\ref{laplace}) (respetando el orden en que aparecen): $a_k$ (el coeficiente que acompaña a $t_{k-1, j}$), $b_k$, $c_k$, $d_k$ (el coeficiente que acompaña a $t_{k, j-1}$ y $t_{k, j+1}$). Notar que por la observación anterior estos coeficientes sólo dependen del radio.



\begin{equation}
\label{sisecu}
  \left\lbrace
  \begin{array}{l}
     t_{0,0} = b_0 \\
     t_{0,1} = b_1 \\
     \vdots\\
     t_{0,n-1} = b_{n-1} \\
		 a_1 t_{0,0} + b_1 t_{1,0} + c_1 t_{2, 0} + d_1 t_{1, n-1} + d_1 t_{1, 1} = 0\\
		 \vdots\\
		 a_1 t_{0,n-1} + b_1 t_{1,n-1} + c_1 t_{2, n-1} + d_1 t_{1, n-2} + d_1 t_{1, 0} = 0\\
		 a_2 t_{1,0} + b_2 t_{2,0} + c_2 t_{3, 0} + d_2 t_{2, n-1} + d_2 t_{2, 1} = 0\\
		 \vdots\\
		 a_{m-1} t_{m-2,n-1} + b_{m-1} t_{m-1,n-1} + c_{m-1} t_{m, n-1} + d_{m-1} t_{m-1, n-2} + d_{m-1} t_{m-1, 0} = 0\\
		 t_{m,0} = b_{n\times m}\\
		 \vdots\\
		 t_{m, n-1} = b_{n\times (m+1)-1}
		 
  \end{array}
  \right.
\end{equation}




\subsection{Estructuración del código}
Para el modelado del problema diseñamos dos módulos: Matriz y Sistema. \\
El primero provee una representación para matrices con una interfaz conveniente. Utilizamos como representación interna un vector de vectores, junto con la cantidad de filas y columnas de la matriz. Las operaciones de Matriz permiten, dado un vector de términos independientes, obtener la solución de un sistema triangular superior sin ceros en la diagonal (o triangular inferior sin ceros en la diagonal) mediante \textit{backward substitution} (respectivamente \textit{forward substitution}), realizar eliminación gaussiana sin pivoteo (en caso de que sea posible), y, de existir, obtener la factorización LU de una matriz. Adicionalmente, hicimos otras versiones de eliminación gaussiana y factorización LU que requieren que la matriz en la que se aplican sea banda, y aprovechan este hecho para reducir la cantidad de operaciones básicas a realizar.\\

\newpage

\section{Resultados y discusión}
\subsection{Evaluación de los métodos}

\subsubsection{Metodo utilizado para el calculo de la isoterma}

Una vez obtenidas todas las tempraturas del sistema buscamos por cada ángulo del horno entre que dos puntos deberia estar la tempratura buscada de la isoterma. Una vez localizados estos puntos suponemos que el crecimiento de la tempratura es lineal, lo cual no necesariamente es cierto pero si se toman puntos suficientemente cercanos el error es ínfimo, a partir de esta supocición podemos facilmente plantear la ecuación de una recta que pasa por los dos puntos que conocemos y deducir de esto donde deberia estar el punto que tiene la temperatura que nos interesa.

\subsection{Criterios de analisis para la isoterma}
 Para tener una refencia a partir de la cual decidir si el horno podia llegar a estar en peligro o no, decidimos utilizar tres criterios clasicos, el promedio, la mediana y el máximo, ya que cada uno precenta ciertos aspectos utiles. La medio o promedio permite tener una idea general de los valores que tiene la isoterma, permitiendo darnos una idea basica de que tantos ángulos o con que tanta intensidad estan superando el umbral. La mediana permite eliminar outliers, medidas que esten fuera de lugar respecto de las que aparezcan por mayoria no afectaran el resultado final y se obtendra una idea mas clara del valor que se esta teniendo mayormente. El máximo permite ver picos que podrian pasar desapercibidos viendo unicamente la media o mediana.

Una vez escogidos estos criterios de analisis lo que hicimos fue para una instancia de isoterma dada ver si al calcular el promedio, mediana o máximo alguno de estos supera cierto umbral escogido entonces se dira que el horno esta en peligro. Para elegir el umbral se aconseja basarse en casos previos de hornos de caracteristicas similares que sufrieron daños, a partir de esto estudiar la isoterma en esos casos con los criterios establecidos previamente y deducir valores adecuados dentro de los cuales sea recomendable trabajar.

\subsubsection{Relación entre granularidad y precision en el calculo de la isoterma}

A través de una serie de experimentos buscamos estudiar que factores contribuyen a una mayor precisión en el cálculo de la posición de la isoterma. Logrando así una predicción mas fiable del peligro en el que puede llegar a estar el horno y sin perder el tiempo con calculos innecesarios.


Al realizar los experimentos nos interesamos en estudiar como la granularidad afectaba la detección precisa de la isoterma, para asi poder estar seguros si el horno estaba o no en peligro. Para hacer esto separamos los experimentos en dos, primero estudiamos que pasaba cuando utilizabamos una mayor cantidad de angulos y luego lo mismo para los radios.

\subsubsection{Granularidad de los ángulos}

A través de nuestros experimentos pudimos observar como al aumentar la cantidad de ángulos se podian detectar mejor los cambios bruscos en la isoterma, permitiendo ver con mayor claridad donde comienzan y donde acaban los picos mientras mejor sea la granularidad. En caso de utilizar una granularidad pobre se ven picos poco precisos o incluso graficos de isotermas erroneos que fallan en detectar el pico, lo cual podria conllevar consecuencias muy graves al no advertir un posible peligro en el horno.

En este primer experimento se utiliza una instancia en la cual la temperatura en todos los ángulos externos es igual excepto en una pequeña zona donde aumenta.

\begin{figure}[H]
\centering
\begin{minipage}{0.48\textwidth}
  \centering
    \includegraphics[width=1\textwidth]{imgs/comp_angulos/comp_angs_iso0.png}
  \caption{\footnotesize{Con 300 ángulos se puede detectar a la perfección un pico en la isoterma.}}
  \label{fig:Experimento1}
\end{minipage}%
\hspace{0.03\textwidth}
\begin{minipage}{0.48\textwidth}   
  \centering
    \includegraphics[width=1\textwidth]{imgs/comp_angulos/comp_angs_iso3.png} 
  \caption{\footnotesize{Con 10 ángulos si bien se detecta un pico, la forma que se muestra está muy alejada de la real, esto podría causar falsas alarmas o la ausencia de las mismas.}}
  \label{fig:Experimento2}
\end{minipage}
\end{figure}


\begin{figure}[H]
\begin{minipage}{0.48\textwidth}   
  \centering
    \includegraphics[width=1\textwidth]{imgs/comp_angulos/comp_angs_iso4.png} 
  \caption{\footnotesize{Con 8 ángulos se puede observar como el pico desaparece fallando dramaticamente la predicción en la forma de la isoterma}}
  \label{fig:Experimento3}
\end{minipage}
\end{figure}

\subsubsection{Granularidad de los radios}
Al estudiar las posibilidades en la cantidad de radios a utilizar se puede observar como a mayor granularidad mejora la posición de la isoterma, convergiendo a la posición real. A diferencia de los ángulos, con los radios los picos serán detectados sin tomar demasiadas precauciones pero para asegurar que el tamaño de los picos, y de la isoterma en general, sean predichos de forma confiable se recomendara utilizar una cantidad de radios adecuadamente alta según la necesidad que se tenga.

Para verificar un poco esto utilizamos dos instancias de prueba. La primera forma una isoterma con forma de óvalo debido a un aumento en la temperatura externa en dos zonas opuestas. La segunda es un círculo perfecto producido por una temperatura constante a lo largo de todo el interior y exterior del horno. En ambos casos veremos como la figura va cambiando su tamaño al tender a la isoterma real.

\subsubsection{Instancia 1}

\begin{figure}[H]
\centering
\begin{minipage}{0.48\textwidth}
  \centering
    \includegraphics[width=1\textwidth]{imgs/comp_rads_bueno/comp_radss_iso5.png}
  \caption{\footnotesize{Con 30 radios consigue una aproximación muy buena del tamaño de la isoterma.}}
  \label{fig:Radios1}
\end{minipage}%
\hspace{0.03\textwidth}
\begin{minipage}{0.48\textwidth}   
  \centering
    \includegraphics[width=1\textwidth]{imgs/comp_rads_bueno/comp_radss_iso0.png} 
  \caption{\footnotesize{Con 4 radios el tamaño de la isoterma esta muy alejado del real.}}
  \label{fig:Radios2}
\end{minipage}
\end{figure}

\subsubsection{Instancia 2}

  \footnotesize{Sucede lo mismo que en en el experimento anterior, al tomar una cantidad relativamente alta de radios se logra un error muy pequeño del tamaño de la isoterma, en este caso se utilizaron 60. La diferencia entre utilizar 40, 50 y 60 radios no fue muy grande, al menos para nuestros estandares, por lo cual se podrian utilizar simplemente 40 consiguiendo un buena relacion entre tiempo de cálculo y aproximación al tamaño de la isoterma, para apreciar mejor esto pueden acceder a http://bit.ly/1Uqt90x donde alojamos algunas animaciones de nuestros experimentos.}

\begin{figure}[H]
\centering
\begin{minipage}{0.48\textwidth}
  \centering
    \includegraphics[width=1\textwidth]{imgs/comp_rads_malo/comp_rads_iso5.png}
	\caption{Se utilizaron 60 radios}  
  \label{fig:Radios3}
\end{minipage}%
\hspace{0.03\textwidth}
\begin{minipage}{0.48\textwidth}   
  \centering
    \includegraphics[width=1\textwidth]{imgs/comp_rads_malo/comp_rads_iso0.png} 
	\caption{Se utilizaron 9 radios} 
  \label{fig:Radios4}
\end{minipage}
\end{figure}

\newpage
\subsection{Evaluación de los métodos}

\subsubsection{Único $b$}
Nuestro primer análisis de los métodos utilizados para la resolución del problema va a consistir en una teórica. 

La resolución mediante el método de eliminación Gaussiana tiene dos partes, la primera es la eliminación gaussiana propiamente dicha, que tiene un costo de $O(k^3)$ flops, donde $k$ es la cantidad de filas (y columnas) de la matriz y la segunda es el algoritmo llamado \emph{backwards substitution}, que tiene un costo de $O(k^2)$ flops.

La resolución mediante el método de factorización LU tiene tres partes, la primera es obtener la factorización LU de la matriz en cuestión, que tiene un costo de $O(k^3)$, y luego aplicar \emph{forward substitution} y \emph{backward substitution}, cada uno con un costo de $O(n^2)$ flops.



Para medir los tiempos de ambas implementaciones decidimos hacer dos experimentos separados, sin embargo similares. 
En uno fijamos una cantidad de radios y movemos la cantidad de angulos, y en el otro al revés.
Lo hicimos así para que las dimensiones de la matriz (ancho o alto, dado que son iguales porque es cuadrada) crezca linealmente, ya que si hacemos variar ángulos y radios al mismo tiempo, las dimensiones dejan de crecer linealmente.
Queremos que las dimensiones crezcan linealmente para que los resultados se entiendan mejor, dado que una escala lineal es, generalmente, más naturales a la vista.


\begin{figure}[H]
\centering
\begin{minipage}{0.48\textwidth}
  \centering
    \includegraphics[width=1\textwidth]{imgs/tiempos_vanilla_angulos.png}
  \caption{\footnotesize{Tiempo tomado por la nuestra implementación de eliminación gaussiana y de factorización LU para resolver el problema. La granularidad de radios está fija en 40 y la de ángulos se indica en el eje $x$. La barra principal indica el promedio, y el segmento indica la desviación standard.}}
  \label{fig:tiempo1}
\end{minipage}%
\hspace{0.03\textwidth}
\begin{minipage}{0.48\textwidth}   
  \centering
    \includegraphics[width=1\textwidth]{imgs/tiempos_vanilla_radios.png} 
  \caption{\footnotesize{Tiempo tomado por la nuestra implementación de eliminación gaussiana y de factorización LU para resolver el problema. La granularidad de ángulos está fija en 40 y la de radios se indica en el eje $x$. La barra principal indica el promedio, y el segmento indica la desviación standard.}}
  \label{fig:tiempo2}
\end{minipage}
\end{figure}



Como se ve en las figuras \ref{fig:tiempo1} y \ref{fig:tiempo2}, las performances entre las implementaciones son realmente similares. También puede observarse algo que nos llamó la atención, que es que el método por eliminación gaussiana tuvo consistentemente peor performance que LU, contradiciendo parcialmente el análisis a priori. Sin embargo, la diferencia es mínima: siempre menor al 3\%.

Por esta razón no creemos que sea algo importante a tener en cuenta, y podríamos atribuirlo a optimizaciones del compilador, dado que los códigos son realmente parecidos (de hecho hicimos nuestra implementación de la factorización LU sobre nuestra implementación de eliminación gaussiana), y las partes que son diferentes le agregan complejidad a la factorización LU.

Por eso lo atribuimos a cuestiones relacionadas con la optimización en tiempo de compilación y no a errores en las mediciones, dado que los tests fueron corridos varias veces por el hecho de que estos resultados eran llamativos.


Por otro lado, vemos que dividir entre ángulos y radios no hace diferencia en la performance, dado que como esperabamos, el runtime solamente depende del tamaño de la matriz, es decir $n (m+1)$, que en ambos gráficos es igual columna a columna.


Ahora es el turno de las implementaciones que aprovechan que la matriz es banda.
La diferencia es enorme. Para comparar rápidamente, podemos usar una tabla:


\begin{figure}[H]
\centering
\begin{tabular}{|r | c  c  c  c|}
\hline
  \textbf{Implementación} & Gauss & Gauss Banda & LU & LU Banda\\ \hline
  \textbf{Tiempo (segundos)} & 26.24 & 0.98 & 25.53 & 1.04 \\
\hline
\end{tabular}

  \caption{\footnotesize{Tiempo promedio tomado por las implementaciones para resolver un sistema con $n = m+1 = 40$.}}
  \label{fig:tiempocomp}
\end{figure}

Con la Figura \ref{fig:tiempocomp} se observa como la diferencia entre las implementaciones \emph{vanilla} y las optimizadas es abismal, superando el 2600\%.

Esta tabla no pretende analizar las implementaciones optimizadas, simplemente probar la diferencia de rendimiento que se obtiene.

A continuación analizaremos las implementaciones optimizadas.

[ACA ANALIZAMOS LA COMPLEJIDAD O LO REPETIMOS CORTITO]

\begin{figure}[H]
\centering
\begin{minipage}{0.48\textwidth}
  \centering
    \includegraphics[width=1\textwidth]{imgs/tiempos_opt_angulos.png}
  \caption{\footnotesize{Tiempo tomado por la nuestra implementación optimizada de eliminación gaussiana y de factorización LU para resolver el problema. La granularidad de radios está fija en 40 y la de ángulos se indica en el eje $x$. La barra principal indica el promedio, y el segmento indica la desviación standard.}}
  \label{fig:tiempoopt1}
\end{minipage}%
\hspace{0.03\textwidth}
\begin{minipage}{0.48\textwidth}   
  \centering
    \includegraphics[width=1\textwidth]{imgs/tiempos_opt_radios.png} 
  \caption{\footnotesize{Tiempo tomado por la nuestra implementación optimizada de eliminación gaussiana y de factorización LU para resolver el problema. La granularidad de ángulos está fija en 40 y la de radios se indica en el eje $x$. La barra principal indica el promedio, y el segmento indica la desviación standard.}}
  \label{fig:tiempoopt2}
\end{minipage}
\end{figure}


Los resultados que se ven en las Figuras \ref{fig:tiempoopt1} y \ref{fig:tiempoopt2} son muy interesantes. Primero notemos que como los procesos de eliminación gaussiana y factorización LU ambos consumen menos tiempo, se nota más la ventaja que le saca gauss a LU con un solo $b$, dado que los procesos de substitución empiezan a pesar asintóticamente.


Como la complejidad de los algoritmos optimizados es $O(\text{radios}^3   \text{angulos})$, en los experimentos se refleja que es mucho más caro aumentar la granularidad con respecto a los radios que con respecto a los ángulos.

\subsubsection{Múltiples $b$'s}




\newpage

\section{Conclusiones}
Para empezar, queremos decir que creemos que este trabajo práctico fue fructífero. Pudimos implementar correctamente los métodos numéricos de resolución de sistemas lineales vistos en clase, lo cual nos ayudó mucho a entenderlos e internalizarlos mucho más.

Además fue muy positiva la experiencia de utilizar estos métodos para resolver un problema de la vida real. Este TP, por otro lado, nos permitió darnos un poco de idea de cuáles son las aplicaciones más usuales de los algoritmos vistos en la materia. Otra cosa notable que se desprende de este TP es lo importante que es abarcar un problema interdisciplinariamente: en este caso necesitamos conocer la ecuación de Laplace (física), saber que una derivada se puede aproximar con un cociente incremental con diferencias finitas (matemática) y utilizar métodos numéricos para resolver sistemas de ecuaciones lineales (computación).

En la misma línea, tomar tiempos de los algoritmos y comparar sus performances nos permitió corroborar lo que ya sabíamos en la teoría de manera práctica, además de ganar intuición sobre el comportamiento de los algoritmos.

Una cosa que nos gustaría decir, es que los experimentos que llegaron al TP son sólo una pequeña proporción de todos los que realizamos. Muchos quedaron afuera porque estaban mal hechos (pocas instancias de prueba, bugs) o simplemente porque no mostraban de forma tan clara lo que queriamos expresar. De manera iterativa, llegamos a los experimentos que presentamos que, desde nuestro punto de vista, expresan perfectamente lo que queríamos que expresaran.

Pasando a las conclusiones sobre la experimentación propiamente dicha, pudimos determinar empíricamente las siguientes cuestiones:
\begin{itemize}
\item Una mayor granularidad de los ángulos que se toman en la discretización favorece a detectar picos localizados de temperaturas que podrían pasar desapercibidos de lo contrario. A su vez, una mayor granularidad de los radios permite delinear con mucha mayor precisión la isoterma. Sin embargo, como vimos, a medida que aumentamos la cantidad de radios la posición de la isoterma va convergiendo a una posición que varía cada vez menos.
\item Aumentar la granularidad de la discretización aumenta también el tiempo de cómputo requerido para resolver el sistema de ecuaciones. Para el caso en que sólo se pasa una instancia, los métodos de eliminación gaussiana y de factorización LU muestran una performance casi idéntica, mientras que en el caso en que se toman múltiples instancias vemos como la factorización LU supera ampliamente a la eliminación gaussiana, confirmando lo que se esperaba desde la teoría (la complejidad de eliminación gaussiana es cúbica para resolver cada instancia, mientras que LU tiene un costo amortizado cuadrático). Por otro lado, las versiones optimizadas que aprovechan el hecho de que la matriz sea banda demostraron ser por mucho superiores a sus contrapartes tradicionales.
\item Por los dos puntos anteriores, es posible encontrar una granularidad \emph{óptima} (en algún sentido) tal que la isoterma obtenida sea lo suficientemente precisa sin pagar un costo en tiempo de cómputo excesivo, pues dijimos que al aumentar los radios la posición de la misma convergía, llegando un punto en el que un aumento de la granularidad produce una difertencia mínima en el resultado.
 \end{itemize}

Para finalizar, nos gustaría plantear trabajo a futuro, y atar algunos cabos sueltos. Nuestra implementación \emph{optimizada} de los algoritmos de eliminación gaussiana y factorización LU que aprovechaban el hecho de que la matriz es banda, solamente tenía una ganancia de performance con respecto al tiempo, dado que seguimos almacenando la matriz en un vector de vectores.

Por esa razón, es interesante plantearse el problema de como llevar a cabo esta optimización de la complejidad espacial de nuestros métodos. Algunas ideas en lo que respecta a la solución de este problema pueden hallarse en \cite[Cap. 4.3]{golub}. 





\newpage

\section{Apéndices}
\subsection{Demo}
Primero observemos que A es una matriz diagonal dominante no estricta. Para eso tenemos que ver que para cada fila el valor absoluto de la diagonal es mayor o igual que la norma-1 del resto de los elementos de esa fila. En nuestro caso puntual, esto significa ver que 
$|c_k| \geq |a_k| + 2|b_k| + |d_k|$\footnote{Los coeficientes están definidos en la sección \ref{sec:armado-sistema}}.

Primero calculemos el lado derecho de la desigualdad:
\begin{equation*}
\left\vert \dfrac{r_k - \Delta r}{r_k (\Delta r)^2}\right\vert +
2 \times \left\vert \dfrac{1}{r_k^2(\Delta \theta)^2} \right\vert+
\left\vert \dfrac{1}{(\Delta r)^2} \right\vert = 
\dfrac{\left\vert r_k - \Delta r \right\vert}{r_k (\Delta r)^2} +
2 \times \dfrac{1}{r_k^2(\Delta \theta)^2} +
\dfrac{1}{(\Delta r)^2}
\end{equation*}

Asumamos primero que $\left\vert r_k - \Delta r \right\vert \geq 0$. En ese caso la ecuación anterior es igual a
 
\begin{center}
$\dfrac{2 r_k^2 (\Delta \theta)^2 - (\Delta r) r_k (\Delta \theta)^2 + 2 (\Delta r)^2}{r_k^2 (\Delta r)^2 (\Delta \theta)^2}$
\end{center}

Entonces, queremos ver que $|b_k|$ es mayor o igual que eso, es decir

\begin{equation*}
\dfrac{\left\vert -2 r_k^2 (\Delta \theta)^2 + (\Delta r) r_k (\Delta \theta)^2 - 2 (\Delta r)^2 \right\vert } 
{\left\vert r_k^2 (\Delta r)^2 (\Delta \theta)^2 \right\vert } \geq
\dfrac{2 r_k^2 (\Delta \theta)^2 - (\Delta r) r_k (\Delta \theta)^2 + 2 (\Delta r)^2}{r_k^2 (\Delta r)^2 (\Delta \theta)^2}
\end{equation*}

Que es equivalente a

\begin{equation*}
\left\vert -2 r_k^2 (\Delta \theta)^2 + (\Delta r) r_k (\Delta \theta)^2 - 2 (\Delta r)^2 \right\vert \geq
2 r_k^2 (\Delta \theta)^2 - (\Delta r) r_k (\Delta \theta)^2 + 2 (\Delta r)^2
\end{equation*}

Supongamos que lo que está dentro del módulo es positivo, entonces tenemos

\begin{center}
$-2 r_k^2 (\Delta \theta)^2 + (\Delta r) r_k (\Delta \theta)^2 - 2 (\Delta r)^2 \geq
2 r_k^2 (\Delta \theta)^2 - (\Delta r) r_k (\Delta \theta)^2 + 2 (\Delta r)^2$ \\
$\Updownarrow$\\
$2\times (-2 r_k^2 (\Delta \theta)^2 + (\Delta r) r_k (\Delta \theta)^2 - 2 (\Delta r)^2) \geq 0$\\
$\Updownarrow$\\
$-2 r_k^2 (\Delta \theta)^2 + (\Delta r) r_k (\Delta \theta)^2 - 2 (\Delta r)^2 \geq 0$
\end{center}

Pero esta última desigualdad vale pues habíamos supuesto que efectivamente eso era positivo.

Ahora veamos que pasa si lo de adentro del módulo es negativo. Tenemos que 

\begin{center}
$-2 r_k^2 (\Delta \theta)^2 + (\Delta r) r_k (\Delta \theta)^2 - 2 (\Delta r)^2 \leq
-2 r_k^2 (\Delta \theta)^2 + (\Delta r) r_k (\Delta \theta)^2 - 2 (\Delta r)^2$ \\
$\Updownarrow$\\
$0 \leq 0$\\
\end{center}

Que vale trivialmente. Luego probamos que la matriz del sistema es diagonal dominante no estricta, si $~{\left\vert r_k - \Delta r \right\vert \geq 0}$. Notar además que por la última cuenta no es cierto que sea diagonal dominante estricta.
\end{document}
