\subsection{Evaluación de los métodos}

\subsubsection{Relación entre la granularidad y precision en el calculo de la isoterma}

A traves de una serie de experimentos buscamos estudiar que factores en el calculo de la isoterma contribuyen a dar mayor precision y cuales no son relevantes para asi lograr hacer una predicción mas fiable del peligro en el que puede llegar a estar el horno y sin perder el tiempo con calculos innecesarios. Para esto primero tuvimos que decidir cual seria el criterio con el que analizariamos si el horno estaba en peligro o no, decidimos utilizar el promedio, la mediana y el maximo ya que cada uno precentava ciertas ventajas. Con estos, analizabamos la isoterma y si se pasaba cierto umbral deciamos que estaba en peligro.


Una vez planteado esto se hizo evidente que mientras mayor sea la cantidad de angulos del sistema mejor iba a ser la aproximacion de la isoterma real por lo que lograriamos detectar mejor ciertos picos y tenerlos en cuenta para poder decidir si el horno estaba en peligro o no.

\subsubsection{Granularidad de los angulos}

\begin{figure}[H]
\centering
\begin{minipage}{0.48\textwidth}
  \centering
    \includegraphics[width=1\textwidth]{imgs/Img_isoterma/comp_angulos0_iso.png}
  \caption{\footnotesize{Con 300 angulos se puede detectar a la perfeccion un pico en la isoterma, esto deja de suceder cuando la cantidad de angulos disminuyen.}}
  \label{fig:Experimento1}
\end{minipage}%
\hspace{0.03\textwidth}
\begin{minipage}{0.48\textwidth}   
  \centering
    \includegraphics[width=1\textwidth]{imgs/Img_isoterma/comp_angulos5_iso.png} 
  \caption{\footnotesize{Se puede observar como en este caso el pico desaparece y la prediccion de la isoterma comienza a ser muy pobre}}
  \label{fig:Experimento2}
\end{minipage}
\end{figure}


\subsubsection{Granularidad de los radios}
